\chapter{État de l'art}
% Définir les termes : route, routeur, restful, bgp, trou noir

\section{Définitions des termes principaux}

\subsection{Routeur}
C'est un élément intermédiaire permettant le transit des paquets d'un réseau à un autre réseau.

\subsection{Route}
C'est un chemin,permettant de relier une source à une destination. Le routeur possède dans une table(table de routage) la correspondance entre son adresse et l'adresse de destination du paquet qu'il a reçu.

\subsection{Restful}
REST(REpresentational State Transfer) est un principe d'architecture qui s'applique aux services Web. Le serveur et le client communiquent sans que le client ne connaisse les informations stockées sur le serveur. Il utilise des requêtes HTTP un peu modifiées, PUT pour modifier ou mettre à jour l'état d'un objet, GET pour récupérer un objet, POST pour créer un objet et DELETE pour supprimer un objet.

\subsection{BGP}
Le protocole BGP, Border Gateway Protocol, est un protocole de routage. il permet l'échange des informations contenu dans un réseau avec un autre réseau.

\subsection{Trou noir ou Black hole}
Ce dit d'une adresse IP fictive. Le routeur qui renvoie les données vers cette adresse IP, les enverra vers Null. Ce point fera donc disparaître le trafic.

\section{Outils principaux}

\subsection{ExaBGP}

\subsection{Nemu}